% Begin CJK
\usepackage{xltxtra,fontspec,xunicode}

%% \CJKsetecglue{\hskip 0.15em plus 0.05em minus 0.05em}
%% slanfont: 允许斜体
%% boldfont: 允许粗体
%% CJKnormalspaces: 仅忽略汉字之间的空白,但保留中英文之间的空白。
%% CJKchecksingle: 避免单个汉字单独占一行。
\usepackage[slantfont, boldfont]{xeCJK}

% Chinese line break strategy
\XeTeXlinebreaklocale "zh"
\XeTeXlinebreakskip = 0pt plus 1pt minus 0.1pt

% 设置中文标点样式,支持quanjiao、banjiao、kaiming等多种方式
\punctstyle{kaiming}

% Default Chinese font
\setCJKmainfont[BoldFont={Adobe Heiti Std}, ItalicFont={Adobe Kaiti Std}]{Adobe Song Std}
% Chinese sans font
\setCJKsansfont[BoldFont={Adobe Heiti Std}]{Adobe Kaiti Std}
% Chinese mono font
\setCJKmonofont{Adobe Heiti Std}

% English sarif font
\setmainfont{DejaVu Serif}
% English sans font
\setsansfont{DejaVu Sans}
% English mono font
\setmonofont{DejaVu Sans Mono}

% CJK fonts
\setCJKfamilyfont{song}{Adobe Song Std}
\setCJKfamilyfont{kai}{Adobe Kaiti Std}
\setCJKfamilyfont{hei}{Adobe Heiti Std}
\setCJKfamilyfont{fangsong}{Adobe Fangsong Std}
\setCJKfamilyfont{lisu}{LiSu}
\setCJKfamilyfont{youyuan}{YouYuan}

%% 自定义宋体
\newcommand{\song}{\CJKfamily{song}}
%% 自定义楷体
\newcommand{\kai}{\CJKfamily{kai}}
%% 自定义黑体
\newcommand{\hei}{\CJKfamily{hei}}
%% 自定义仿宋体
\newcommand{\fangsong}{\CJKfamily{fangsong}}
%% 自定义隶书
\newcommand{\lisu}{\CJKfamily{lisu}}
%% 自定义幼圆
\newcommand{\youyuan}{\CJKfamily{youyuan}}

% Begin mathematical envs
\newtheorem{example}{例}
\newtheorem{problem}{问题}
\newenvironment{proof}{\noindent\textbf{证明\ }}{\hspace*{\fill}$\Box$\medskip}
\newenvironment{solution}{\noindent\textbf{解答\ }}{\hspace*{\fill}$\Box$\medskip}
\newtheorem{algo}{算法}

\newtheorem{definition}{定义}
\newtheorem{axiom}{公理}
\newtheorem{theorem}{定理}
\newtheorem{lemma}{引理}
\newtheorem{corollary}{推论}

\newtheorem{property}{性质}
\newtheorem{proposition}{命题}
\newtheorem{remark}{注解}
\newtheorem{condition}{条件}
\newtheorem{assumption}{假设}
\newtheorem{conclusion}{结论}

% Section title
\renewcommand{\contentsname}{目录}
\renewcommand{\abstractname}{摘要}
\renewcommand{\indexname}{索引}
\renewcommand{\listfigurename}{插图目录}
\renewcommand{\listtablename}{表格目录}
\renewcommand{\figurename}{图}
\renewcommand{\tablename}{表}
\renewcommand{\algorithmcfname}{算法}
\renewcommand{\appendixname}{附录}
\renewcommand{\appendixpagename}{附录}
\renewcommand{\appendixtocname}{附录}
\renewcommand\refname{参考文献}
% \renewcommand\bibname{参考文献}

% Section title format
\titleformat{\chapter}{\centering\huge}{第\thechapter{}章}{1em}{\textbf}
\titleformat{\section}{\centering\sihao}{\thesection}{1em}{\textbf}
\titleformat{\subsection}{\xiaosi}{\thesubsection}{1em}{\textbf}
\titleformat{\subsubsection}{\xiaosi}{\thesubsubsection}{1em}{\textbf}
