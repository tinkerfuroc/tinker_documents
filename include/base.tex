
% Geometry package, setting margin.
\usepackage[top=2.5cm, bottom=2.5cm, left=2.5cm, right=2.5cm]{geometry}
\usepackage{multicol}

% Enum control.
\usepackage{enumerate}

% Hyperlink
\usepackage[
    pdfstartview=FitH,
    CJKbookmarks=true,
    bookmarks=true,
    bookmarksnumbered=true,
    bookmarksopen=true,
    colorlinks=true,
    citecolor=black,
    linkcolor=black,
    anchorcolor=black,
    urlcolor=black
]{hyperref}

% Title format
\usepackage{titlesec}
% List of figure, table, etc.
\usepackage{titletoc}
% Table control
\usepackage{booktabs}
% Appendix control
\usepackage[title,titletoc,header]{appendix}

% Font size control
\usepackage{type1cm}
% Indentation control
\usepackage{indentfirst}
\usepackage{changepage}

% Color control
\usepackage{color,xcolor}

%% AMS LaTeX
\usepackage{amsmath,amssymb}
\usepackage{latexsym,textcomp}

%% By Xiao Hanyu
%% 数学公式中的黑斜体
% \usepackage{bm}

%% By Xiao Hanyu
%% 调整公式字体大小:\mathsmaller, \mathlarger
% \usepackage{relsize}

% graphics
\usepackage{graphicx}
\usepackage{subfig}
\usepackage{indentfirst}
\usepackage{float}
\usepackage{url}

% pgf/tikz graph
\usepackage{pgf,tikz}
\usetikzlibrary{shapes,automata,snakes,backgrounds,arrows}
\usetikzlibrary{mindmap}

\usepackage{fancyhdr}
\pagestyle{plain}

%% By Xiao Hanyu
%% 有时会出现\headheight too small的warning
\setlength{\headheight}{15pt}

% Source code
\usepackage{listings}
% Algorithm
\usepackage[ruled,vlined]{algorithm2e}
% Tree graph
\usepackage{synttree}

\usepackage{multirow}

% Source code settings
\lstloadlanguages{}
\lstset{
showstringspaces=false,              %% 设定是否显示代码之间的空格符号
numbers=left,                        %% 在左边显示行号
numberstyle=\tiny,                   %% 设定行号字体的大小
basicstyle=\footnotesize,            %% 设定字体大小\tiny, \small, \Large等等
keywordstyle=\color{blue!70}, commentstyle=\color{red!50!green!50!blue!50},
                                     %% 关键字高亮
frame=shadowbox,                     %% 给代码加框
rulesepcolor=\color{red!20!green!20!blue!20},
escapechar=`,                        %% 中文逃逸字符,用于中英混排
xleftmargin=2em,xrightmargin=2em, aboveskip=1em,
breaklines,                          %% 这条命令可以让LaTeX自动将长的代码行换行排版
extendedchars=false                  %% 这一条命令可以解决代码跨页时,章节标题,页眉等汉字不显示的问题
}
