\documentclass[a4paper,12pt]{article}
\usepackage{xeCJK}
\usepackage{xunicode,xltxtra}
\usepackage{amssymb,latexsym}
\usepackage{fancyhdr}
\usepackage{geometry}
\usepackage{setspace}
\usepackage{amsmath}
\usepackage{booktabs}
\usepackage{caption}
\usepackage{indentfirst}

\setmainfont{Times New Roman}
\setCJKmainfont{微软雅黑}
\setCJKmonofont{微软雅黑}

\renewcommand{\figurename}{图}

\makeatletter
\newcommand\dlmu[2][3cm]{\hskip1pt\underline{\hb@xt@ #1{\hss#2\hss}}\hskip3pt}
\makeatother

\topmargin=-0.45in      %
\evensidemargin=0in     %
\oddsidemargin=0in      %
\textwidth=6.5in        %
\textheight=9.0in       %
\headsep=0.25in         %

\pagestyle{fancy}
\renewcommand\headrulewidth{0.4pt}
\renewcommand\footrulewidth{0.4pt}
\rfoot{Page\ \thepage\ of\ \protect\pageref{LastPage}} 
\makeatletter %双线页眉
\def\headrule{{\if@fancyplain\let\headrulewidth\plainheadrulewidth\fi%
		\hrule\@height 1.0pt \@width\headwidth\vskip1pt%上面线为1pt粗
		\hrule\@height 0.5pt\@width\headwidth  %下面0.5pt粗
		\vskip-2\headrulewidth\vskip-1pt}      %两条线的距离1pt
	\vspace{6mm}}     %双线与下面正文之间的垂直间距

\author{董渊$\:$自43\\2014011493}
\title{实验三\ 组合逻辑电路的设计 \\
	\begin{flushright}
		\begin{large}
			%单个\ 表示空格呦~~~
			——电子技术实验终结报告 \ \  第九周周一
		\end{large}
	\end{flushright}}
\date{2015/10/12 }

\begin{document}
\begin{spacing}{1}
\maketitle
\end{spacing}

\section{苍新点}
\subsection{迷之物体识别}
在采用SRC之前,我本来是通过计算两个矩阵的标准化降采样矩阵的点击来判断相似性;这样做的结果是点积值没有什么实际意义,对于不同的物体相差非常大,因此难以取合适的阈值,所以判断的鲁棒性不高。在换成SRC之后,鲁棒性有了显著的提升,同时速度和存储空间上并没有显著的降低。

\section{末端视觉}
\subsection{原始图片的获得}
tk2\_com/com\_firmware的节点arm\_webcam将读取来自于一个640*480分辨率的logitech webcam的图片,并且利用ros的image\_transport将图片包装为一个类型为sensor\_msgs::Image的消息,并且将其以10Hz的频率发布在主题tk2\_com/arm\_cam\_image下,供后序模块读取。
\subsection{前景与后景的分离,以及物体的分离}
tk2\_vision/arm\_target\_finder的节点targetfinder将接收tk2\_com/arm\_cam\_image的消息。

正如上一个模块,深度视觉,所说的那样,对于室内场景的图片,我们近似的有如下的先验知识:
\begin{enumerate}
	\item 场景内总存在大平面:例如地面或者墙面。可见一般来说,这样子的大平面都是后景。
	\item 在光照条件良好的情况下,大平面的颜色和纹理近似一致。这意味着大平面也就是后景的熵非常低,可以近似的看作纯色物块。所以,在本实验的情况下,由于被机械手抓起来的物件是小物件,所以小物件作为前景,一般因颜色复杂且与背景不同而熵非常高。这意味着,我们可以通过对熵做一个给定阈值的二值化,来将大平面滤去。
	\item 由于我们的摄像头水平安装在机械臂的手心,所以我们的摄像头拍摄到的图片是垂直于地面的;而一般来说,即使是在复杂的室内环境中,物体由于重力的原因,几乎总是接触一个与地面平行的平面;所以,从垂直于地面的方向看去,在水平于地面的方向上,组成同一个物体的像素数目在水平轴上的分布应当是连续的;这个数目的突变意味着已经到达物体的边沿。
\end{enumerate}
因此,我将以如下的方式实现该算法:
\begin{enumerate}
	\item 计算图片的熵所成的矩阵。
	
	对于一张给定的图片,其熵的定义如下:
	\begin{equation}
		E = -\Sigma p_{i}\cdot log_{2}p_{i}
	\end{equation}
	
	首先,我们对图片进行灰度化;接下来,对于图片的每个点,我们将用这个点上下左右各扩展filter\_size\_长度所成的正方形图片的熵来填充这个点;对于边沿的点,则以直接填充0代替。由于这个矩阵只是用来做蒙板,所以可以适当降低采样来大幅提高运算速度;这里x轴和y轴的采样都被降低到了原来的四分之一,在不影响实际效果的情况下,速度提高了十六倍,对整个算法的运算速度有极其显著的影响。
	
	\item 对熵矩阵进行二值化
	
	通过对各种实际情况的测试,我们最终选择了0.265作为阈值。这个阈值相对来说是比较松的,不会导致物体因为熵过低而被整个除去,但是后果就是物体在光照下的阴影也会被读进来成为了“物体”的一部分。
	
	\item 对上一步获得的图片进行“打开”操作
	
	这一步是为了将由上一步导致的,图片由于中间部分熵低两边部分熵高,而导致部分地区出现分段,打开图片可以有效除去这些分断,而将整张图片连到一起。
	
	\item 识别图片中的边沿,并且按照边沿所围成的图片面积大小,将较小的边框全部删除。
	
	这时候所留下来的部分就是物体。
	
	\item 依水平方向统计各列数目,依照突变所在列分段。一般来说,物体部分比阴影部分高;另一方面,有些特别的物体(比如柜子的边沿)将远远超出物体的高度,并且在水平方向上非常窄。所以,我们除去特别矮的和远远高出一般值的物体轮廓,剩下来的就是我们感兴趣的物体轮廓。将这个轮廓作为前景蒙板,从而获得我们感兴趣的,并且已经相互分离的物体的实际图像,被滤去的后景以纯黑色像素点填充。
\end{enumerate}
\subsection{基于SRC的图像识别}
在上一个模块中,我们已经将物体从后景中剥离出来并且已经分离;不妨直接取所有物体中轮廓最大的那个作为当前我们研究的物体,这是因为在边沿的物体由于只能拍到一部分,所以必然比在中间的物体小。

我们将基于SRC,对上述的图像进行识别。

\paragraph{SRC的原理}
首先,我们事先准备一定数量的,该物体的样本;我们做如下的先验假设,即,当样本数量足够时,我们可以认为,每张该物体的照片,都可以近似表示为该物体的样本的照片的线性叠加,并且与其他物体的样本的照片线性无关,这也就是“稀疏表示”的含义。写作式子的话,也就是
\begin{equation}\label{srcwithoutnoise}
\boldsymbol{y} = A\boldsymbol{x_{0}} 
\end{equation}
,其中$\boldsymbol{x_{0}} = [0, ..., 0, a_{i, 1}, ..., a_{i, n_{i}}, 0, ..., 0]$,如果该物体属于第i类。

一般来说,系统$\boldsymbol{y} = A\boldsymbol{x}$都是欠定的,所以并没有唯一解。解L2范数最小化获得的$\hat{x_{2}}$一般是稠密的,这意味着我们无法从$\hat{x_{2}}$中迅速识别出物体属于哪一类。 而L0范数最小化是个NP-hard的问题,甚至连估算数值解都非常困难;所以我们用L1范数最小化问题来近似:
\begin{equation}\label{l1}
(\ell^{1})\qquad\hat{x_{1}} = \text{arg}\ \text{min} ||\boldsymbol{x} ||_{1} \qquad\text{s.t.}\: \boldsymbol{y} = A \boldsymbol{x} 
\end{equation}

事实上,由于噪声的存在,我们无法精确的保证我们的样本满足\eqref{srcwithoutnoise}的要求;所以我们修正\eqref{srcwithoutnoise},来容许可能的小范围误差:
\begin{equation}\label{srcwithnoise}
\boldsymbol{y} = A\boldsymbol{x_{0}} + \boldsymbol{z}
\end{equation}
其中,$\boldsymbol{z}$是一个能量有界的误差,也即$||\boldsymbol{z}||_{2} < \epsilon$。在这种情况下,稀疏解$x_{0}$依然可以近似的被如下的“稳定L1范数最小化问题”所估算:
\begin{equation}\label{stable_l1}
(\ell^{1}_{s})\qquad\hat{x_{1}} = \text{arg}\ \text{min} ||\boldsymbol{x} ||_{1} \qquad\text{s.t.}\: ||A\boldsymbol{x}-\boldsymbol{y}||_{2} \leq \epsilon 
\end{equation}
这个凸优化问题是能被有效解决的;并且可以证明$\hat{x_{1}}$能充分接近$\hat{x_{0}}$。

所以,算法Sparse Representation-based Classification(即SRC)的实现方法如下:

首先给定一波样本矩阵A,被测试样例$\boldsymbol{y}$,以及可容忍误差$\epsilon$;

将A标准化使其有单位的L2范数,然后解\eqref{stable_l1};

依次计算残差$r_{i}(\boldsymbol{y}) = ||\boldsymbol{y}-A\delta_{i}(\hat{x_{1}})||_{2}$,其最小值即为我们识别的及诶过。

\paragraph{图片特征提取}
为了降低$\boldsymbol{y}$的维数,从而加快运算速度,对图片进行特征提取是很有必要的;从数学上可以证明,只要提取的特征向量的维度大于一个O(lgn)的上界,就不会影响\eqref{stable_l1}的解,因此我们直接对图片进行降采样即可。


\section{参考文献}
J. Wright, A. Y. Yang, A. Ganesh, S. S. Sastry, and Y. Ma, “Robust Face Recognition via Sparse Representation”, IEEE Transactions On Pattern Analysis and Machine Intelligence(PAMI), vol. 31, no. 2, pp. 210-227, 2009.

\label{LastPage}
\end{document}